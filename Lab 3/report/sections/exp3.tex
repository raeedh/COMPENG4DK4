\section*{Experiment 3}
% Cellular networks are often designed so that under busy traffic conditions, the highest acceptable probability of blocking is about 1% (i.e., the probability of rejecting a cell-phone call is 1% because the cellular network has no available channels.) Generate a graph of maximum offered loading (in Erlangs) versus the number of cellular channels needed to achieve this blocking probability performance.
The maximum offered loading (in Erlangs) versus the number of cellular channels needed to achieve a 1\% blocking probability is graphed on Figure~\ref{fig:exp3}. The minimum number of cellular channels required to have an "acceptable" cellular network with less than 1\% blocking probability is 5 cellular channels, which remains acceptable for a maximum offered load of 1 Erlang. The increase in the maximum offered loading follows demonstrates mostly linear growth.

\begin{figure}[htp]
\centering
\begin{tikzpicture}
	\begin{axis}[
		title = {Maximum Offered Loading vs. Number of Cellular Channels},
		width = \textwidth,
		%height = 150,
		xmin = 1, xmax = 40,
		ymin = 0, ymax = 30,
		ylabel = Maximum Offered Loading (Erlangs), xlabel =Number of Cellular Channels,
		xticklabel style={
		/pgf/number format/fixed,
		/pgf/number format/precision=3
		},
		scaled x ticks=false,
		% some code for adding points
		%nodes near coords={%
		%\footnotesize
		%$(\pgfmathprintnumber
		%{\pgfkeysvalueof{/data point/x}},
		%\pgfmathprintnumber
		%{\pgfkeysvalueof{/data point/y}})$%
		%},
	]
	\addplot table [y=$off_load$, x=$num_chan$]{./data/exp3.dat};
	\end{axis}
\end{tikzpicture}
\caption{Experiment 3}
\label{fig:exp3}
\end{figure}