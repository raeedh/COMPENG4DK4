\section*{Experiment 6}
% There is a packet switching link that is servicing both real-time voice and best-effort data traffic. The data traffic arrives according to a Poisson process as in Part 2, but where service times are exponentially distributed with a mean service time of 40 ms. There is also a voice traffic stream encoded using the G.711 (64 Kbps) voice codec. In this case, packets arrive periodically with fixed inter-packet arrival times equal to tv = 20 ms. The packets include 62 bytes of header overhead in addition to the voice payload. Assume that the transmission link operates at 1 Mbps.
% Write a simulation for this situation assuming that all arriving packets are placed in the same buffer and served in FCFS order. The best way to do this is to generate two new arrival events for the two streams (voice and data). Generate results which show the mean delay of the voice and data packets versus the data packet arrival rate (i.e., MEAN ARRIVAL RATE). Plot a separate curve for each voice stream delay and data packet delay on the same graph.

The packet length of each G.711 encoded voice packet was determined to be 1792 bits per packet, or 224 bytes per packet. As the voice packets arrive every 20 ms, each voice packet must contain 20 ms of the voice payload. Therefore, each packet contains a 62 byte header and 160 bytes of the voice payload (20 ms at 64 Kbps = 1280 bits or 160 bytes).