\section*{Experiment 2}
% Generate a plot of mean delay (in msec) vs PACKET ARRIVAL RATE in the same fashion as in Lab 1. Set PACKET LENGTH to 500 bits and assume that the output link operates at a bit rate of 1 Mbps (i.e., LINK BIT RATE). Make sure that you include runs using your McMaster student ID number as the seed.

A plot of the mean delay vs. packet arrival rate is shown in Figure~\ref{fig:exp2}. At low packet arrival rate values, we see the mean delay approaches 0.5 msec. The mean delay axis intercept at these low packet arrival rates equal to the packet length divided by the link bit rate, in this experiment this is 500 bits divided by 1000 bits per msec, or 0.5 msec. Similar to Lab 1, we can observe that the mean delay begins to increase exponentially from this mean delay axis intercept value as we begin to increase the packet arrival rate.


\begin{figure}[htp]
\centering
\begin{tikzpicture}
	\begin{axis}[
		title = {Mean Delay vs. \texttt{PACKET\_ARRIVAL\_RATE}},
		width = \textwidth,
		%height = 150,
		xmin = 0, xmax = 2000,
		ymin = 0, ymax = 10.5,
		ylabel = Mean Delay (msec), xlabel = \texttt{PACKET\_ARRIVAL\_RATE},
		xticklabel style={
		/pgf/number format/fixed,
		/pgf/number format/precision=3
		},
		scaled x ticks=false,
		% some code for adding points
		%nodes near coords={%
		%\footnotesize
		%$(\pgfmathprintnumber
		%{\pgfkeysvalueof{/data point/x}},
		%\pgfmathprintnumber
		%{\pgfkeysvalueof{/data point/y}})$%
		%},
	]
	\addplot table [y=$mean_delay$, x=$arr_rate$]{./data/exp2.dat};
	\end{axis}
\end{tikzpicture}
\caption{Experiment 2: Mean Delay vs. Packet Arrival Rate}
\label{fig:exp2}
\end{figure}