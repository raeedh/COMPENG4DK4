\section*{Experiment 3}
% Consider a packet switch where arrivals to be transmitted on an outgoing link can be modelled as a Poisson process with an arrival rate of λ packets/sec. When the link is busy transmitting, arriving packets are stored in a large buffer at the switch. The packets are drawn from the buffer in FCFS (first-come-first-served) order and transmitted non-preemptively, i.e., the transmission of a packet is always completed once it starts. Assume that all packets have a fixed length of 2000 bits each and that the output link operates at 1 Mbps. Modify the simulation and use it to find the maximum value of λ such that the probability that a packet’s delay exceeds 20 msec is less than 2%.

The code was modified to add a check to see if the delay for any packet is greater than 20 msec. The function used to check this is shown in Listing~\ref{list:exp3}. The modified code was run with different values of \texttt{PACKET\_ARRIVAL\_RATE} $\lambda$ to determine the maximum value of $\lambda$ where the average probability was less than 2\%. The maximum value determined as 401 packets per second, with the probability of a packet's delay exceeds 20 msec beginning to exceed 2\% at 402 packets per second.
% 401: 1.93% of packets > 20 msec delay
% 402: 2.01% of packets > 20 msec delay

\begin{lstlisting}[gobble=8, language=C, caption={Modifications to Experiment 3 Code}, label={list:exp3}][h]
	void check_delay(Simulation_Run_Ptr simulation_run, int arr[]) {
		Simulation_Run_Data_Ptr data;
	
		data = (Simulation_Run_Data_Ptr) simulation_run_data(simulation_run);
	
		// Get current packet and time
		double a = 1e3 * data->accumulated_delay;
		int b = data->number_of_packets_processed;
	
		// Check if packet has changed
		if (arr[1] + 1 == b) {
			// Check if change in time is greater than 20 (packet delay > 20 msec)
			if (a > arr[2] + 20) arr[0]++;
			// Update with new packet and time
			arr[1] = b;
			arr[2] = a;
		}
	}
\end{lstlisting}