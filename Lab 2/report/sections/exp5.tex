\section*{Experiment 5}
% There are three packet switches connected into a network as shown below. Each of the switches has local packet arrivals at rates of λ1, λ2 and λ3 packets/second, as shown. The packet arrival process can be modelled as Poisson and all packets are 1000 bits. At each switch, the arriving packets are buffered in FCFS order, awaiting transmission on the outgoing link. All of the packets that arrive at Switch 1 are transmitted over Link 1, which is a wireless link that Switches 2 and 3 are listening on. Each packet that is transmitted on Link 1 is destined for Link 2 with probability p12 = 1 − p13, independently for each transmitted packet. Eventually, all arriving packets are transmitted over either Link 2 or 3. The transmission bit rates are 2 Mbps, 1 Mbps and 1 Mbps for Links 1, 2 and 3, respectively. Create a simulation that models the performance of this network. Fix λ2 and λ3 to the same value of 500 packets/second, and set λ1 to 750 packets/second. Generate plots of mean delay vs. p12, showing the mean delay experienced by packets originating at Switches 1, 2 and 3, on the same graph. The total delay of a packet is from the time it arrives to one of the switches, until it is fully transmitted on either Link 2 or 3.

A plot of the mean delay vs. $p_{12}$ is shown in Figure~\ref{fig:exp5}.

\begin{figure}[htp]
    \centering
    \begin{tikzpicture}
        \begin{axis}
            [
            title = {Mean Delay vs. \texttt{$p_{12}$}},
            width = \textwidth,
            %height = 150,
            xmin = 0, xmax = 1.0,
            ymin = 0, ymax = 3.2,
            ylabel = Mean Delay (msec), xlabel = \texttt{$p_{12}$},
            xticklabel style={
                /pgf/number format/fixed,
                /pgf/number format/precision=3
            },
            scaled x ticks=false,
            % some code for adding points
            %nodes near coords={%
            %\footnotesize
            %$(\pgfmathprintnumber
            %{\pgfkeysvalueof{/data point/x}},
            %\pgfmathprintnumber
            %{\pgfkeysvalueof{/data point/y}})$%
            %},
            ]
            \addplot table [y=$mean_delay$, x=$P_{12}$]{./data/exp5.dat};
        \end{axis}
    \end{tikzpicture}
    \caption{Experiment 5: Mean Delay vs. $p_{12}$}
    \label{fig:exp5}
\end{figure}