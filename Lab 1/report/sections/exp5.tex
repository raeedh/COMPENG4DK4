\section*{Experiment 5}
% Using Queueing Theory it can be shown that the mean delay for a work conserving single server queueing system with Poisson process arrivals and a general service time distribution (i.e., an M/G/1 queue) is given by2 dM/G/1 = X + λX2 2(1 − ρ) (2) where X is the mean service time (SERVICE TIME), ρ is the product of the mean arrival rate and the mean service time, i.e., ρ = λX, λ = ARRIVAL RATE, and X2 = σ2 X +  ̄X2 is the second moment of the service time3. Note that the total delay of a customer is the sum of its service time and its queueing delay. Therefore, the two terms in Equation 2 are the mean service time and the mean queueing delay incurred by the customers, respectively. When the service times are fixed (i.e., an M/D/1 queue, as in the supplied simulation), σX = 0, and therefore, the mean delay can be written as 4 dM/D/1 =  ̄X(2 − ρ) 2(1 − ρ) Compare the results that you plotted in Parts 2 and 4 to this analytic result.
