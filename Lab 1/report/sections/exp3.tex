\section*{Experiment 3}
% What happens when the product of ARRIVAL RATE and SERVICE TIME, in Expression 1 is greater than 1? To see, set ARRIVAL RATE slightly greater than 1/SERVICE TIME, then do a run of 10,000 customers. What happens when you keep increasing the run length? Explain why is this happening. Explain why the condition in Expression 1 is necessary.
We kept the same value of \texttt{SERVICE\_TIME} and increased the \texttt{ARRIVAL\_RATE} to 0.2, such that the product of \texttt{SERVICE\_TIME} $\times$ \texttt{ARRIVAL\_RATE} $= 2$. At 10,000 customers served the mean delay has increased to over 50,000, significantly more than largest value seen when the arrival rate was $<0.1$.
% What happens when you keep increasing the run length?
As we increase the run length (number of customers to serve), the mean delay begins to increase proportionally with the number of customers to serve as seen in Table~\ref{table:exp3arr2}.
% Explain why is this happening.
As the system becomes saturated, every customer entering the system will enter a queue and will need to wait a mean delay that is proportional to the number of customers already waiting in queue (each additional customer increases the mean delay by the same amount).
% Explain why the condition in Expression 1 is necessary.
The condition in Expression 1 is necessary to prevent such a scenario where increasing the number of customers will affect the mean delay, unlike the case where the arrival rate satisfies Expression 1. For example, we can see in Table~\ref{table:exp3arr5} that the mean delay is not affected by the number of customers to serve. Satisfying this condition also ensures that we are able to empty out the queue and reach a steady state, otherwise the server will never be able to catchup as more customers arrive than the server can service.

\begin{table}[h]
\centering
\begin{tabular}{|l|l|}\hline
	Number of Customers to Serve & Mean Delay    \\\hline
	10000                        & 50160.191256  \\\hline
	20000                        & 99962.602135  \\\hline
	50000                        & 249997.485025 \\\hline
	100000                       & 499914.794732 \\\hline
\end{tabular}
\caption{Experiment 3: Mean Delay when Arrival Rate $= 0.2$}
\label{table:exp3arr2}
\end{table}

\begin{table}[h]
	\centering
	\begin{tabular}{|l|l|}\hline
		Number of Customers to Serve & Mean Delay \\\hline
		10000                        & 14.989001  \\\hline
		20000                        & 14.989132  \\\hline
		50000                        & 14.982623  \\\hline
		100000                       & 14.972480  \\\hline
	\end{tabular}         
	\caption{Experiment 3: Mean Delay when Arrival Rate $= 0.05$}
	\label{table:exp3arr5}
\end{table}