\section*{Experiment 6}
% Modify the program so that instead of M/D/1 you simulate an M/M/1 queuing system, i.e., instead of Poisson process arrivals and fixed service times, we now have Poisson process arrivals and exponentially distributed service times with the same mean of SERVICE TIME i.e., When the program needs a service time, exponential generator((double)SERVICE TIME); will return an exponentially distributed service time with the proper mean5. Repeat Part 2 above using the same value of SERVICE TIME. Plot the mean delay vs. ARRIVAL RATE for the M/D/1 and M/M/1 cases on the same graph. Explain the differences and similarities. Repeat Part 5, using Equation 2 for an M/M/1 system. Give a simple expression for dM/M/1.

\begin{figure}[h]
	\centering
	\begin{tikzpicture}
		\begin{axis}[
			title = {Mean Delay vs. \texttt{ARRIVAL\_RATE}},
			width = 0.98\textwidth,
			%height = 400,
			xmin = 0, xmax = 0.1,
			ymin = 0, ymax = 1050,
			ylabel = Mean Delay, xlabel = \texttt{ARRIVAL\_RATE},
			xticklabel style={
			/pgf/number format/fixed,
			/pgf/number format/precision=3
			},
			scaled x ticks=false,
			yticklabel style={
			/pgf/number format/fixed,
			/pgf/number format/precision=3
			},
			scaled y ticks=false,
			legend pos=north west,
			% some code for adding points
			%nodes near coords={%
			%\footnotesize
			%$(\pgfmathprintnumber
			%{\pgfkeysvalueof{/data point/x}},
			%\pgfmathprintnumber
			%{\pgfkeysvalueof{/data point/y}})$%
			%},
		]
		\addplot table [y=$mean_delay$, x=$arr_rate$]{../data/exp2.dat};
		\label{plot:exp6mmi}
		\addlegendentry{M/D/I}
		\addplot table [y=$mean_delay$, x=$arr_rate$]{../data/exp6.dat};
		\label{plot:exp6mdi}
		\addlegendentry{M/M/I}
		\end{axis}
	\end{tikzpicture}
	\caption{Experiment 6: Mean Delay vs. Arrival Rate}
	\label{fig:exp6}
	\end{figure}