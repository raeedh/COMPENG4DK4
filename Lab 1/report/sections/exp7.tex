\section*{Experiment 7}
% Modify the code so that it simulates a single server queueing system with a maximum queue size, e.g., define MAX_QUEUE_SIZE at the top of the simulation. When a customer arrives and finds that the number in the queue is at its maximum, the arriving customer is rejected from the system without being served. At the end of each simulation run, output the customer rejection probability, i.e., the ratio of the number of rejected customers to the total number of customer arrivals. Note that only served customers are included in the mean delay calculation. In this type of system, there is no limit on the arrival rate that can be used, since customers that find a full queue will simply be rejected, i.e., the queue can never grow larger than MAX_QUEUE_SIZE. For this reason, the upper limit in Expression 1 need not be satisfied. For various values of MAX_QUEUE_SIZE, plot the mean customer delay versus arrival rate. Make sure that you use values of arrival rate that are sufficiently large to see the asymptotic value of mean delay. Also do plots of the rejection fraction versus the same arrival rate values. Discuss the differences and similarities between the infinite and finite queue cases. Find a relationship between the mean delay asymptote and the other system parameters as the arrival rate becomes large.
For arrival rates where the upper limit in Expression 1 is satisfied, the results are the same between the infinite and finite queue cases, except for when the queue length is very small (1 in our experiment).
In this case, the queue acts as a slight bottleneck, and we see that the rejection fraction is non-zero (although still very small).
When the arrival rate upper limit is not satisfied, the infinite queue has a much higher mean delay for those arrival rates, compared to the finite queue.
The mean delay asymptote value as the arrival rate becomes large, is the \texttt{SERVICE_TIME} * \texttt{MAX_QUEUE_SIZE}.
