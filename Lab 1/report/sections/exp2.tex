\section*{Experiment 2}
% Make sure that SERVICE TIME is set to a value of 10. Then vary ARRIVAL RATE over the range 0 < ARRIVAL RATE × SERVICE TIME < 1 (1) doing simulation runs to collect results. For each change in the simulation you must recompile and run the program1. For each run, record the results printed on the screen when the run finishes. For each value of ARRIVAL RATE, you should do several simulation runs (at least 10) using different random number generator seeds. The random number seeds are set by the RANDOM SEED preprocessor macro near the top of the program. Use your McMaster ID number as the random seed for one of the runs at each value of ArrivalRate. The length of each run (i.e., NUMBER TO SERVE) should be very large, e.g., millions of customers (the more the better). 
% Generate a plot of mean delay vs. ARRIVAL RATE. The value for mean delay that you plot should be the average of the values you obtain for different random generator seeds at a particular value of ARRIVAL RATE. 
% Explain the reason for the mean delay value at low ARRIVAL RATE values, i.e., What is the mean delay axis intercept in terms of the other system parameters? Explain the behaviour of the mean delay as the ARRIVAL RATE increases and approaches the right-hand inequality in Expression 1. What is the value of the ARRIVAL RATE axis asymptote when this happens? Explain the shape of the mean delay curve obtained.
If we set \texttt{SERVICE\_TIME} to a value of 10, we need to set $0 < \text{\texttt{ARRIVAL\_RATE}} < 0.1$ to satisfy Expression~\ref{eq:eq1}:
\begin{equation}
	0 < \text{\texttt{ARRIVAL\_RATE}} \times \text{\texttt{SERVICE\_TIME}} < 1
	\label{eq:eq1}
\end{equation}
We selected the following values for \texttt{ARRIVAL\_RATE}: 0.001, 0.01, 0.03, 0.05, 0.07, 0.09, 0.095, 0.099. We doubled the variable \texttt{NUMBER\_TO\_SERVE} from the default to \num{50e6} to \num{100e6}. We modified the provided code to loop through each arrival rate with all 18 random number generator seeds, and print out the average results of the seeds for each arrival rate.
% Generate a plot of mean delay vs. ARRIVAL RATE.
A plot of the mean delay vs. arrival rate is shown in Figure~\ref{fig:exp2}.

% Explain the reason for the mean delay value at low ARRIVAL RATE values, i.e., What is the mean delay axis intercept in terms of the other system parameters?
At low arrival rate values, we see the mean delay approaches 10. The mean delay axis intercept at these low arrival rates is the service time, because customers will have very short queue times when the arrival rate is low. Therefore the majority of their delay will become the service time, as queue times become shorter as arrival rates decrease.
% Explain the behaviour of the mean delay as the ARRIVAL RATE increases and approaches the right-hand inequality in Expression 1. What is the value of the ARRIVAL RATE axis asymptote when this happens?
As the arrival rate increases and approaches the right-hand inequality in Expression~\ref{eq:eq1}, the mean delay begins to increase exponentially. As the arrival rate increases, there are on average more customers in the system at any time, and customers must wait in queue for longer before being serviced. The value of the arrival rate axis asymptote is 0.1, which is the upper limit of \texttt{ARRIVAL\_RATE} that satisfies Expression~\ref{eq:eq1} when the value of \texttt{SERVICE\_TIME} is set to 10.
% Explain the shape of the mean delay curve obtained.
The mean delay curve that is obtained is an exponential curve where the mean delay increases slowly at low values of the arrival rate but begins to increase significantly as the arrival rate reaches the upper bound of the expression.

\begin{figure}[h]
\centering
\begin{tikzpicture}
	\begin{axis}[
		title = {Mean Delay vs. \texttt{ARRIVAL\_RATE}},
		width = \textwidth,
		%height = 150,
		xmin = 0, xmax = 0.1,
		ymin = 0, ymax = 550,
		ylabel = Mean Delay, xlabel = \texttt{ARRIVAL\_RATE},
		xticklabel style={
        /pgf/number format/fixed,
        /pgf/number format/precision=3
		},
		scaled x ticks=false,
		% some code for adding points
		%nodes near coords={%
		%\footnotesize
		%$(\pgfmathprintnumber
		%{\pgfkeysvalueof{/data point/x}},
		%\pgfmathprintnumber
		%{\pgfkeysvalueof{/data point/y}})$%
		%},
	]
	\addplot table [y=$mean_delay$, x=$arr_rate$]{../data/exp2.dat};
	\end{axis}
\end{tikzpicture}
\caption{Experiment 2: Mean Delay vs. Arrival Rate}
\label{fig:exp2}
\end{figure}