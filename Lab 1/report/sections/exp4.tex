\section*{Experiment 4}
% Repeat Part 2 but this time set the SERVICE TIME to 30 and vary ARRIVAL RATE over a suitable range. Compare the mean delay vs. ARRIVAL RATE curve to what you obtained before. Plot this curve on a graph with the curve from Part 2. Explain the differences in the curves.
When setting the \texttt{SERVICE\_TIME} to 30, we similarly adjusted our \texttt{ARRIVAL\_RATE} values by dividing the rates used in experiment 2 by 3. We kept all other parameters the same (including the random number generator seeds) and reran the simulation. A plot of the mean delay vs. arrival rate of both experiments 2 and 4 are shown in Figure~\ref{fig:exp4}. We can see that in this experiment, with a larger service time the mean delay increases significantly faster at the same arrival rates. The baseline mean delay at low arrival rates has also increased to match the increased service time.

\begin{figure}[h]
\centering
\begin{tikzpicture}
	\begin{axis}[
		title = {Mean Delay vs. \texttt{ARRIVAL\_RATE}},
		width = 0.98\textwidth,
		%height = 400,
		xmin = 0, xmax = 0.1,
		ymin = 0, ymax = 1600,
		ylabel = Mean Delay, xlabel = \texttt{ARRIVAL\_RATE},
		xticklabel style={
		/pgf/number format/fixed,
		/pgf/number format/precision=3
		},
		scaled x ticks=false,
		yticklabel style={
		/pgf/number format/fixed,
		/pgf/number format/precision=3
		},
		scaled y ticks=false,
		legend pos=north east,
		% some code for adding points
		%nodes near coords={%
		%\footnotesize
		%$(\pgfmathprintnumber
		%{\pgfkeysvalueof{/data point/x}},
		%\pgfmathprintnumber
		%{\pgfkeysvalueof{/data point/y}})$%
		%},
	]
	\addplot table [y=$mean_delay$, x=$arr_rate$]{../data/exp2.dat};
	\label{plot:exp4exp2}
	\addlegendentry{Experiment 2}
	\addplot table [y=$mean_delay$, x=$arr_rate$]{../data/exp4.dat};
	\label{plot:exp4exp4}
	\addlegendentry{Experiment 4}
	\end{axis}
\end{tikzpicture}
\caption{Experiment 4: Mean Delay vs. Arrival Rate}
\label{fig:exp4}
\end{figure}