\section*{Experiment 6}

To simulate this simulation, we re-used the simulation from Part 5 as the S-ALOHA reservation channel, and created an additional channel for data transmission. When transmission successfully ends on the reservation channel, we schedule a start event for data packet transmission, which is nearly identical to the existing packet transmission function \texttt{schedule\_transmission\_start\_event}. This eventually leads to an end transmission event for data packets, similar to the transmission functions already provided. The data packet end transmission event is shown in Listing~\ref{list:exp6}.

\begin{lstlisting}[language=c,caption=Data Packet Transmission End Event, label=list:exp6]
void data_transmission_end_event(Simulation_Run_Ptr simulation_run, void *packet) {
	Packet_Ptr this_packet, next_packet;
	Buffer_Ptr data_buffer;
	Time now;
	Simulation_Run_Data_Ptr data;
	Channel_Ptr channel;

	data = (Simulation_Run_Data_Ptr) simulation_run_data(simulation_run);
	channel = data->data_channel;

	now = simulation_run_get_time(simulation_run);

	this_packet = (Packet_Ptr) packet;
	data_buffer = data->data_channel_queue;

	/* This station has stopped transmitting. */
	decrement_transmitting_stn_count(channel);
	set_channel_state(channel, IDLE);

	TRACE(printf("Success.\n"););

	/* Collect statistics. */
	data->number_of_packets_processed++;

	(data->stations + this_packet->station_id)->packet_count++;
	(data->stations + this_packet->station_id)->accumulated_delay += now - this_packet->arrive_time;

	data->number_of_collisions += this_packet->collision_count;
	data->accumulated_delay += now - this_packet->arrive_time;
	output_blip_to_screen(simulation_run);

	/* This packet is done. */
	free((void *) packet);

	/* See if there is another packet in the data channel queue. If so, enable it for transmission. We will transmit immediately. */
	if (fifoqueue_size(data_buffer) > 0) {
		next_packet = fifoqueue_get(data_buffer);

		schedule_data_transmission_start_event(simulation_run, now, (void *) next_packet);
	}
}
\end{lstlisting}

The mean delay versus number of stations curve can be seen in Figure~\ref{fig:exp6_ns}. The mean data packet duration, $X$, was set to 1 second, the packet arrival rate was set to 0.05, and the slot duration, $X_r$, was set to 0.1 seconds. The number of stations has no effect on the mean delay of the data packets in the system.

The mean delay versus mean data packet duration curve can be seen in Figure~\ref{fig:exp6_mdpd}. The number of stations was set to 10 stations, the packet arrival rate was set to 0.05, and the slot duration, $X_r$, was set to 0.1 seconds. The mean delay experiences exponential increase as the mean data packet duration increases.

The mean delay versus packet arrival rate curve can be seen in Figure~\ref{fig:exp6_par}. The number of stations was set to 10 stations, the mean data packet duration, $X$, was set to 1 second, and the packet arrival rate was set to 0.05. The mean delay experiences exponential increase as the slot duration, $X_r$, increases.

The mean delay versus packet arrival rate curve can be seen in Figure~\ref{fig:exp6_xr}. The number of stations was set to 10 stations, the mean data packet duration, $X$, was set to 1 second, and the slot duration, $X_r$, was set to 0.1 seconds. The mean delay experiences exponential increase as the mean data packet duration increases.


\begin{figure}[h]
\centering
\begin{tikzpicture}
	\begin{axis}[
		title = {Mean Delay vs. \texttt{NUMBER\_OF\_STATIONS}},
		width = 0.98\textwidth,
		height = 200,
		xmin = 1, xmax = 15,
		ymin = 0, ymax = 2,
		ylabel = Mean Delay, xlabel = \texttt{NUMBER\_OF\_STATIONS},
		xticklabel style={
		/pgf/number format/fixed,
		/pgf/number format/precision=3
		},
		scaled x ticks=false,
		% some code for adding points
		%nodes near coords={%
		%\footnotesize
		%$(\pgfmathprintnumber
		%{\pgfkeysvalueof{/data point/x}},
		%\pgfmathprintnumber
		%{\pgfkeysvalueof{/data point/y}})$%
		%},
	]
	\addplot table [y=$delay_ns$, x=$ns$]{./data/exp6_ns.dat};
	\end{axis}
\end{tikzpicture}
\caption{Experiment 6: Mean Delay vs. Number of Stations}
\label{fig:exp6_ns}
\end{figure}

\begin{figure}[h]
\centering
\begin{tikzpicture}
	\begin{axis}[
		title = {Mean Delay vs. \texttt{MEAN\_DATA\_PACKET\_DURATION}},
		width = 0.98\textwidth,
		% height = 200,
		xmin = 0, xmax = 20,
		ymin = 0, ymax = 400,
		ylabel = Mean Delay, xlabel = \texttt{MEAN\_DATA\_PACKET\_DURATION},
		xticklabel style={
		/pgf/number format/fixed,
		/pgf/number format/precision=3
		},
		scaled x ticks=false,
		% some code for adding points
		%nodes near coords={%
		%\footnotesize
		%$(\pgfmathprintnumber
		%{\pgfkeysvalueof{/data point/x}},
		%\pgfmathprintnumber
		%{\pgfkeysvalueof{/data point/y}})$%
		%},
	]
	\addplot table [y=$delay_mdpd$, x=$mdpd$]{./data/exp6_mdpd.dat};
	\end{axis}
\end{tikzpicture}
\caption{Experiment 6: Mean Delay vs. Mean Data Packet Duration, $X$}
\label{fig:exp6_mdpd}
\end{figure}

\begin{figure}[h]
\centering
\begin{tikzpicture}
	\begin{axis}[
		title = {Mean Delay vs. \texttt{PACKET\_ARRIVAL\_RATE}},
		width = 0.98\textwidth,
		% height = 200,
		xmin = 0.001, xmax = 1, xmode = log,
		ymin = 0, ymax = 20,
		ylabel = Mean Delay, xlabel = \texttt{PACKET\_ARRIVAL\_RATE},
		xticklabel style={
		/pgf/number format/fixed,
		/pgf/number format/precision=3
		},
		scaled x ticks=false,
		yticklabel style={
		/pgf/number format/fixed,
		/pgf/number format/precision=1
		},
		scaled y ticks=false,
		% legend pos=north west,
		% some code for adding points
		%nodes near coords={%
		%\footnotesize
		%$(\pgfmathprintnumber
		%{\pgfkeysvalueof{/data point/x}},
		%\pgfmathprintnumber
		%{\pgfkeysvalueof{/data point/y}})$%
		%},
	]
	\addplot table [y=$delay$, x=$arr$]{./data/exp6_par.dat};
	\end{axis}
\end{tikzpicture}
\caption{Experiment 6: Mean Delay vs. Packet Arrival Rate}
\label{fig:exp6_par}
\end{figure}

\begin{figure}[h]
\centering
\begin{tikzpicture}
	\begin{axis}[
		title = {Mean Delay vs. \texttt{SLOT\_DURATION}},
		width = 0.98\textwidth,
		% height = 200,
		xmin = 0, xmax = 2,
		ymin = 0, ymax = 15,
		ylabel = Mean Delay, xlabel = \texttt{SLOT\_DURATION},
		xticklabel style={
		/pgf/number format/fixed,
		/pgf/number format/precision=3
		},
		scaled x ticks=false,
		yticklabel style={
		/pgf/number format/fixed,
		/pgf/number format/precision=1
		},
		scaled y ticks=false,
		% legend pos=north west,
		% some code for adding points
		%nodes near coords={%
		%\footnotesize
		%$(\pgfmathprintnumber
		%{\pgfkeysvalueof{/data point/x}},
		%\pgfmathprintnumber
		%{\pgfkeysvalueof{/data point/y}})$%
		%},
	]
	\addplot table [y=$delay$, x=$xr$]{./data/exp6_xr.dat};
	\end{axis}
\end{tikzpicture}
\caption{Experiment 6: Mean Delay vs. Slot Duration, $X_r$}
\label{fig:exp6_xr}
\end{figure}